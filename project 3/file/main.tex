\documentclass{article}
\usepackage{amsmath}
\usepackage{amsthm}
\usepackage{amssymb}
\usepackage{algorithm}
\usepackage[noend]{algpseudocode}
\usepackage{enumerate}
\usepackage{float}
\usepackage{indentfirst}
\setlength{\parindent}{2em}
\newtheorem{lemma}{Lemma}

\begin{document}

\section{Proof of NP-complete}
To prove that the "Node-Labelling Decision Problem" is NP-complete, we need to show two things:
\begin{enumerate}
    \item The problem is in NP:

    Given a certificate, which is a labeling of the nodes in the graph, we can easily verify in polynomial time whether the labeling is valid or not.
    
    Specifically, given an instance of the Node-Labelling Decision Problem with graph $G=(V,E)$, label set $K={0,1,\dots,k-1}$, and non-negative integer $R$, suppose we have a labeling function $c: V \rightarrow K$. To verify that this labeling is valid and satisfies the requirement that $|C(v, R)| = k$ for every vertex $v \in V$, we can perform the following steps:
    \begin{enumerate}
        \item Check if every label in $K$ is assigned to at least one node in $G$. This can be done by iterating through all vertices $v \in V$ and maintaining a set of encountered labels. This step takes $O(|V|)$ time.

        \item For each vertex $v \in V$, calculate the set $N(v, R)$, which contains all nodes within $R$ hops from $v$. This can be done using a Breadth-First Search (BFS) starting from $v$, and stopping once we reach depth $R$. Since BFS has a time complexity of $O(|V| + |E|)$, this step takes $O(|V|(|V| + |E|))$ time in total for all vertices.

        \item For each vertex $v \in V$, calculate the set $C(v, R)$, which contains the distinct labels within $R$ hops from $v$. This can be done by iterating through $N(v, R)$ and maintaining a set of encountered labels. This step takes $O(|V|^2)$ time in total for all vertices.

        \item Check if $|C(v, R)| = k$ for every vertex $v \in V$. This step takes $O(|V|)$ time.
    \end{enumerate}
    
    The overall time complexity for verifying the solution is $O(|V|^2 + |V|(|V| + |E|))$. Since this is polynomial in the size of the input, we conclude that the Node-Labelling Decision Problem is in NP.
    \item The problem is NP-hard:

    We will show that the problem is NP-hard by reducing the 3-coloring problem (which is known to be NP-complete) to the Node-Labelling Decision Problem. Suppose we have an instance of the 3-coloring problem, which is an undirected graph $G=(V, E)$, and we want to determine if there exists a valid 3-coloring of the graph.

    Now we will construct a new instance $G'=(V',E')$ of the Node-Labelling Decision Problem as follows:
    \begin{enumerate}
        \item Initially, let $V'=V$. (This step takes $O(|V|)$ time to create the set $V'$.)
        \item For each edge $(u, v) \in E$, add a new vertex $w_{(u, v)}$ to $V'$ and edges $(u, w_{(u, v)}), (v, w_{(u, v)})$ to $E'$. (This step takes $O(|E|)$ time to process all edges and create the new vertices and edges.)
        \item For each vertex $u$ in $V$, add two new vertices $u_1$, $u_2$ to $V'$, and add edges $(u, u_1)$, $(u, u_2)$, $(u_1, u_2)$ to $E'$. (This step takes $O(|V|)$ time to process all vertices and create the new vertices and edges.)
        \item Set the set of labels $K = {0, 1, 2}$, and set $R = 1$. (This step takes $O(1)$ time.)
    \end{enumerate}

    The total time complexity of the reduction is $O(|V| + |E|)$, which is polynomial with respect to the size of the input graph $G=(V, E)$. Therefore, the reduction from the 3-coloring problem to the Node-Labelling Decision Problem is in polynomial time.

    Now we will prove that $G$ has a valid 3-coloring if and only if there exists a valid labeling for the Node-Labelling Decision Problem with the constructed $G'$, $K$, and $R$.

    ($\Rightarrow$) Let $c: V \rightarrow K$ be a valid 3-coloring of $G$. We will extend this coloring to the vertices of $G'$ as follows:
    \begin{itemize}
        \item For each vertex $v \in V$, we assign $c'(v) = c(v)$.
        \item For each edge $(u, v) \in E$, we assign $c'(w_{(u, v)}) = 3 - c(u) - c(v)$, which is the remaining color not used by $u$ or $v$.
        \item For each vertex $u \in V$, we assign $c'(u_1) = (c(u) + 1) \mod 3$ and $c'(u_2) = (c(u) + 2) \mod 3$.
    \end{itemize}

    Now, we will show that this labeling $c'$ is a valid solution for the Node-Labelling Decision Problem with $G'$, $K$, and $R$.
    \begin{itemize}
        \item First, all labels in $K$ are used in $G'$, so the labeling is valid.
        \item Second, we need to show that for each vertex $v \in V'$, we have $|C(v, R)| = 3$. Since $R = 1$, we only need to consider the 1-hop neighborhood of each vertex.
        \begin{itemize}
            \item For each vertex $v \in V$, its 1-hop neighbors are $v_1$, $v_2$, and all the $w_{(u, v)}$ where $(u, v) \in E$. Since $c'(v_1) \neq c'(v_2)$ and $c'(v) \neq c'(v_1), c'(v_2)$, we have $|C(v, 1)| = 3$.
            \item For each vertex $w_{(u, v)}$, its 1-hop neighbors are $u$ and $v$. Since $c'(u) \neq c'(v)$ and $c'(w_{(u, v)}) \neq c'(u), c'(v)$, we have $|C(w_{(u, v)}, 1)| = 3$.
            \item For each vertex $u_1$, its 1-hop neighbors are $u$ and $u_2$. Since $c'(u) \neq c'(u_2)$ and $c'(u_1) \neq c'(u), c'(u_2)$, we have $|C(u_1, 1)| = 3$.
            \item Similarly, for each vertex $u_2$, we have $|C(u_2, 1)| = 3$.
        \end{itemize}
    \end{itemize}

    Thus, if $G$ has a valid 3-coloring, there exists a valid labeling for the Node-Labelling Decision Problem with the constructed $G'$, $K$, and $R$.
    
    ($\Leftarrow$) Now, let $c': V' \rightarrow K$ be a valid labeling for the Node-Labelling Decision Problem with $G'$, $K$, and $R$. We will use this labeling to construct a valid 3-coloring for $G$.

    Define a coloring $c: V \rightarrow K$ such that for each vertex $v \in V$, we assign $c(v) = c'(v)$. We will show that $c$ is a valid 3-coloring for $G$.
    
    We need to show that for each edge $(u, v) \in E$, we have $c(u) \neq c(v)$. Recall that we added a vertex $w_{(u, v)}$ in $G'$, and edges $(u, w_{(u, v)})$, $(v, w_{(u, v)})$ in $E'$. Since $c'$ is a valid labeling for the Node-Labelling Decision Problem with $R = 1$, we have $|C(w_{(u,v)}, 1)| = 3$. Therefore, $c'(w_{(u, v)})$, $c'(u)$ and $c'(v)$ must be the three distinct labels. This implies that $c(u) = c'(u) \neq c'(v) = c(v)$.

    Thus, if there exists a valid labeling for the Node-Labelling Decision Problem with the constructed $G'$, $K$, and $R$, then $G$ has a valid 3-coloring.

    In summary, we have shown that $G$ has a valid 3-coloring if and only if there exists a valid labeling for the Node-Labelling Decision Problem with the constructed $G'$, $K$, and $R$. This reduction can be done in polynomial time. Hence, the Node-Labelling Decision Problem is NP-hard.
\end{enumerate}

Since we have shown that the Node-Labelling Decision Problem is both in NP and NP-hard, we conclude that the problem is NP-complete.

\section{Main idea of algorithm}
\begin{enumerate}
    \item Select a root node with the minimum $m$ value using $select\_root\_with\_minimum\_m$ function.
    \item Perform BFS on the tree, starting from the root node.
    \item At each node visited:
    \begin{enumerate}
        \item Compute the impact of each available label using $label\_impact$ function.
        \item Select the label with the minimum impact.
        \item Assign the selected label to the node.
    \end{enumerate}
    \item Continue until all nodes are visited and labeled.
\end{enumerate}

\section{Pseudo code}
\begin{algorithm}[H]
\caption{Tree Node-Labelling Algorithm}
\begin{algorithmic}[1]
\Procedure{ModifiedTreeNodeLabelling}{$G, k$}
    \State $root \gets select\_root\_with\_minimum\_m(G, k)$
    \State Initialize empty dictionary $labels\_assigned$
    \State $BFS\_queue \gets [(root, [])]$
    \While{BFS\_queue is not empty}
        \State $(v, visited) \gets$ dequeue $BFS\_queue$
        \State Compute available labels for node $v$
        \For{each label in available labels}
            \State Compute $label\_impact(v, label)$
        \EndFor
        \State $best\_label \gets$ label with minimum impact
        \State $labels\_assigned[v] \gets best\_label$
        \For{each child in $children\_of(G, v, visited)$}
            \State Enqueue $(child, visited.copy())$ in $BFS\_queue$
        \EndFor
    \EndWhile
    \State \Return $labels\_assigned$
\EndProcedure
\end{algorithmic}
\end{algorithm}

\section{Proof of proximity ratio}
\begin{lemma}
For any node $v$ in the tree, if the label selected by the algorithm results in the smallest $r(v)/m(v)$ ratio for the node, then the overall maximum ratio in the tree is minimized.
\end{lemma}

\begin{proof}
Assume there exists another label assignment that would result in a smaller maximum ratio in the tree. Let $L$ be the set of labels assigned by our algorithm, and $L'$ be the set of labels assigned by the alternative assignment.

For any node $v$, let $r(v, L)$ and $m(v, L)$ be the $r(v)$ and $m(v)$ values when labels are assigned according to $L$, and $r(v, L')$ and $m(v, L')$ be the $r(v)$ and $m(v)$ values when labels are assigned according to $L'$.

Let $v^*$ be the node with the highest $r(v, L)/m(v, L)$ ratio when using our algorithm. Since our algorithm selects the label that results in the smallest $r(v)/m(v)$ ratio for each node, we have:

$r(v^*, L) / m(v^*, L) \leq r(v^*, L') / m(v^*, L')$

Now, let $v'$ be the node with the highest $r(v, L')/m(v, L')$ ratio when using the alternative assignment. Since the maximum ratio in the tree is minimized by our algorithm, we have:

$r(v^*, L) / m(v^*, L) \leq r(v', L') / m(v', L')$

From the two inequalities above, we can conclude that:

$r(v^*, L) / m(v^*, L) \leq r(v', L') / m(v', L') \leq r(v^*, L') / m(v^*, L')$

This contradicts our assumption that there exists another label assignment that would result in a smaller maximum ratio in the tree. Thus, the lemma is proven.
\end{proof}

Now we can analyze the proximity ratio of the algorithm. Let $\rho$ be the maximum ratio achieved by the algorithm for any input instance, and $\rho^*$ be the optimal proximity ratio for that instance.

By Lemma 1, our algorithm minimizes the maximum ratio in the tree. Therefore, $\rho$ must be less than or equal to the maximum ratio of any other possible labeling. Since $\rho^*$ is the optimal proximity ratio for that instance, we know that $\rho^*$ is the smallest possible maximum ratio that can be achieved. Therefore, we have:

$\rho^* \leq \rho$

Now, let's analyze the relationship between $r(v)$ and $m(v)$. Since $r(v)$ must be at least $m(v)$ for any node $v$ (as $m(v)$ is a lower bound of $r(v)$), we know that $r(v) / m(v) \geq 1$. Also, note that the maximum possible $r(v)$ value for any node $v$ is $|V| - 1$, as in the worst case, we would need to visit all nodes in the tree to find all $k$ labels. Therefore, the maximum possible value for $r(v) / m(v)$ is $(|V| - 1) / 1 = |V| - 1$. Thus, we have:

$1 \leq \rho^* \leq \rho \leq |V| - 1$

This inequality establishes that our algorithm has a proximity ratio $\rho$ that is bounded by a constant value for any input instance. While the proximity ratio is not constant across all input instances (as it depends on the size of the graph), it is constant for each individual input instance.

\section{Analysis of Time Complexity}
The time complexity of the algorithm can be analyzed as follows:
\begin{itemize}
  \item The $select\_root\_with\_minimum\_m$ function computes $m(v)$ values for all nodes in the tree. In the worst case, this takes $O(|V|^2)$ time.
  \item The BFS traversal of the tree takes $O(|V| + |E|)$ time, but since the input graph is a tree, $|E| = |V| - 1$, so the BFS traversal takes $O(|V|)$ time.
  \item When visiting each node during BFS, the algorithm computes the impact of each available label using the $label\_impact$ function. Since there are $k$ labels, this step takes $O(k)$ time per node. For all nodes in the tree, this step takes $O(k|V|)$ time.
  \item Selecting the label with the minimum impact and assigning it to the node is an $O(1)$ operation.
  \item Continuing until all nodes are visited and labeled results in a total time complexity of $O(|V|^2) + O(|V|) + O(k|V|) = O(|V|^2 + k|V|)$.
\end{itemize}

As $k$ is a constant, the time complexity of the algorithm is polynomial in the number of nodes, which is $O(|V|^2)$. Thus, the algorithm has polynomial-time complexity.

\end{document}